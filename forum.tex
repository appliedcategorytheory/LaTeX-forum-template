% ===========================
%     PREAMBLE
% ===========================

\documentclass{amsart}          

% ==============================
% PACKAGES
%
% Add your own packages here
% ==============================

\usepackage{amsfonts}
\usepackage{amssymb}  
\usepackage{amsthm} 
\usepackage{amsmath} 
\usepackage{caption}
\usepackage[inline]{enumitem}
\setlist{itemsep=0em, topsep=0em, parsep=0em}
\setlist[enumerate]{label=(\alph*)}
\usepackage{etoolbox}
\usepackage{stmaryrd} 
\usepackage[dvipsnames]{xcolor}
\usepackage[]{hyperref}
\hypersetup{
  colorlinks,
  linkcolor=blue,
  citecolor=blue,
  urlcolor=blue}
\usepackage{graphicx}
\graphicspath{{assets/}}
\usepackage{mathtools}

\usepackage{tikz}
\usetikzlibrary{
  matrix,
  arrows,
  shapes
}

\setcounter{tocdepth}{1} % Sets depth for table of contents. 

% ======================================
% STYLING
%
% For those papers that need tikz (etc.) styles included
%
% ======================================


\usetikzlibrary{decorations.pathmorphing}
\usetikzlibrary{decorations.markings}
\usetikzlibrary{decorations.pathreplacing}
\usetikzlibrary{arrows}
\usetikzlibrary{shapes.geometric}

\pgfdeclarelayer{edgelayer}
\pgfdeclarelayer{nodelayer}
\pgfsetlayers{edgelayer,nodelayer,main}
\tikzstyle{none}=[inner sep=0pt]


% ======================================
% MACROS
%
% Add your own macros below here
% ======================================

% Use \iam for your primary reading response
\newcommand{\iam}[1]{
  \hrule
  \vspace{0.25em}
  \textbf{{#1}'s Reading Response}
  \vspace{0.25em}
  \hrule
  \vspace{1em}
}

% Use \respond to respond to someone's primary response
\newcommand{\respond}[1]{
  \vspace{1em} \textbf{#1}
}


% Draw an (in-line) drawing of the node, given by its tikz style
% e.g. ` \node{P1X}{X} ` draws a node with style given by P1X in ./assets/paper1.tikzstyle, and labelled by `X`
\newcommand{\node}[2]{
	\begin{tikzpicture}[scale=0.3,
    every node/.style={scale=0.75}]
	\begin{pgfonlayer}{nodelayer}
		\node [style=#1] (1)   at (0, 0)   {#2};
		\node [style=none] (2)   at (-1, 0)   {};
		\node [style=none] (3)   at (1, 0)   {};
	\end{pgfonlayer}
	\begin{pgfonlayer}{edgelayer}
		\draw [-] (1.center) to (2.center);
		\draw [-] (1.center) to (3.center);
	\end{pgfonlayer}
	\end{tikzpicture}
}

% Include a tikz figure, found in the ./assets/ folder
% e.g. ` \tikzfig{paper1-CNOT} ` includes the file `./assets/paper1-CNOT.tikz`
% This makes life a lot easier if you need to move folders around.
\newcommand{\tikzfig}[1]{
\InputIfFileExists{#1}{}{\input{./assets/#1.tikz}}
}


% Refer to composition with the command \comp
\newcommand{\comp}{\circ}
% Refer to tensor product with the command \tensor
\newcommand{\tensor}{\otimes}

% bra-kets for quantum computing
\newcommand\bra[1]{\left< \, #1\, \right|}
\newcommand\ket[1]{\left|\, #1 \, \right>}
\newcommand\braket[2]{\left<\, #1 \,|\, #2 \,\right>}

% ======================================
% FRONT MATTER
% ======================================

\author{Forum Participants}
\title{Reading Responses}
\begin{document}
\maketitle{}

% ======================================
% RESPONSES FOR PAPER 0
% Example Reading Response
% ======================================

\section*{Example Reading Response}
\label{sec:ex-response}

\iam{Daniel}

This is an example of how to enter your response. The
content of your response can include anything relevant to
the paper.  Some prompts are
\begin{itemize}
\item thoughts and ideas
\item worked examples
\item what you liked or disliked
\item questions about things you didn't understand
\item questions and thoughts about extentions of the paper.
\end{itemize}
But really feel free to be as creative and out of the box,
as long winded or short winded as you like.  Make
connections to other topics.

Note how I used the \verb|\iam{}| macro to introduce my primary
reading response and see how it appears on the compiled pdf.
If you have any favorite macros, feel free to include them
in the preamble. To keep this document tidy, please place
them in the MACRO's section of the preamble.  If you would
like to include any packages, you can do that in the
``packages'' sections of the preamble.

\respond{Jules} This forum is also for starting
conversations, not just giving one-off responses.  So if
you'd like to respond to someone else, you can use the
\verb|\respond{}| macro for this. Do this to answer
questions posed, ask follow-up questions, react to ideas, or
to respond with anything relevant.

You can also support your claims by citing papers. Write
the bibliography entry at the bottom of the LaTeX file. Cite
it using the command
\begin{verbatim}
    \cite{ENTRY}
\end{verbatim}
where ENTRY is what you named your reference.  Use any
format you like, this forum is an internal document only to
be seen by school participants. If the paper is available
online, link to it for convenience. Note the entry
\verb|\bibitem{CatSem}| in bibilography which, when called,
appears as \cite{CatSem}.

Quick question: what if I want to include an image into a
response?

\respond{Daniel} Even though I'm the original poster, I'm
responding to Jules, so I'll use the \verb|\respond{}|
macro.

To include images, place the file in the \verb|\assets\|
directory then by typing
\begin{verbatim}
   \begin{center}
     \includegraphics[scale=s]{FILE-NAME.XXX}
   \end{center}
\end{verbatim}
where $ 0< s \leq 1 $, FILE-NAME the name of your file, and XXX is the
extension. 

%For example, note the file \verb|groth.jpg| in
%the \verb|\assets\| directory. By including it as stated, you will see
%\begin{center}
%  \includegraphics[scale=0.25]{groth.jpg}
%\end{center}
%in the compiled document.



% =============================
% BIBLIOGRAPHY
%
% Add any citations here.
% Use whatever style you like.
% It's only for us
% =============================


\begin{thebibliography}{99}

\bibitem{CatSem} Bob Coecke, Samson Abramsky. A categorical
  semantics of quantum protocols. \url{https://ieeexplore.ieee.org/abstract/document/1319636} 

\end{thebibliography}

\end{document}
